\small
\begin{tabularx}{\textwidth}{|p{2.5cm}|p{2.5cm}|X|X|X|X|}
\hline
\multicolumn{6}{|X|}{\textbf{Nivel: 01 }}\\\hline\textbf{Materia} & \textbf{Docente} & \textbf{Estudiantes} & \textbf{Aprobados} & \textbf{Promedio} & \textbf{\%Supera el Promedio} \\ \hline
Matemática Básica & BUSTOS VERA RHAY PABLO & 24 & 22 & 37.38 & 50.00 \%\\ \hline
Física y Laboratorio & NEVAREZ TOLEDO MANUEL ROGELIO & 24 & 24 & 37.5 & 54.17 \%\\ \hline
Algoritmos y Pseudocódigo & POSLIGUA FLORES KLEBER ROLANDO & 24 & 23 & 35.33 & 66.67 \%\\ \hline
Algebra Lineal & CARVAJAL CARVAJAL JOSE LUIS & 24 & 21 & 32.92 & 54.17 \%\\ \hline
Tecnologías de la Información  & RIVERA BONE CARLOS ALEJANDRO & 24 & 22 & 43.67 & 75.00 \%\\ \hline
Comunicación Oral y Escrita & CABALLERO MOREIRA JAIRON ADRIAN & 24 & 23 & 39.54 & 50.00 \%\\ \hline
\end{tabularx}

\vspace{1cm}
\section{Análisis de Rendimiento}
El análisis del rendimiento académico de los estudiantes de primer nivel de TICs muestra una variedad de resultados en función de las diferentes materias. En general, los estudiantes han demostrado una buena comprensión en algunas áreas, mientras que otras han requirido un esfuerzo adicional.

En cuanto a las materias específicas, la mayoría de los estudiantes han logrado tener un rendimiento acceptable en Matemática Básica, con un promedio de 37.38 y un porcentaje de aprobados del 50\%. Sin embargo, en Física y Laboratorio, los estudiantes han demostrado una mayor comprensión, con un promedio de 37.5 y un porcentaje de aprobados del 54.17\%.

En cuanto a las materias de programación, Algoritmos y Pseudocódigo ha sido una de las materias más exitosas, con un promedio de 35.33 y un porcentaje de aprobados del 66.67\%. Esto sugiere que los estudiantes han demostrado una buena comprensión de los conceptos fundamentales de programación.

Por otro lado, Algebra Lineal ha sido una de las materias más desafiantes, con un promedio de 32.92 y un porcentaje de aprobados del 54.17\%. Esto puede indicar que los estudiantes necesitan un mayor esfuerzo para comprender los conceptos matemáticos inherentes en esta materia.

En cuanto a las materias de tecnología, Tecnologías de la Información ha sido una de las materias más importantes, con un promedio de 43.67 y un porcentaje de aprobados del 75\%. Esto sugiere que los estudiantes han demostrado una buena comprensión de los conceptos relacionados con la tecnología y la información.

Por último, Comunicación Oral y Escrita ha sido una materia en la que los estudiantes han demostrado un comportamiento más diverso, con un promedio de 39.54 y un porcentaje de aprobados del 50\%. Esto puede indicar que los estudiantes necesitan mejorar su habilidad para comunicar de manera efectiva y clara.

En general, el análisis del rendimiento académico de los estudiantes de primer nivel de TICs muestra que hay áreas en las que los estudiantes han demostrado una mayor comprensión y otras en las que necesitan mejorar. Es importante que los docentes y los estudiantes trabajen juntos para abordar estas necesidades y mejorar el rendimiento académico.\\
\vspace{1cm}\small
\begin{tabularx}{\textwidth}{|p{2.5cm}|p{2.5cm}|X|X|X|X|}
\hline
\multicolumn{6}{|X|}{\textbf{Nivel: 03 }}\\\hline\textbf{Materia} & \textbf{Docente} & \textbf{Estudiantes} & \textbf{Aprobados} & \textbf{Promedio} & \textbf{\%Supera el Promedio} \\ \hline
Redes I & VELASTEGUI IZURIETA HOMERO JAVIER & 16 & 16 & 35.81 & 62.50 \%\\ \hline
Estadística y Probabilidades & BUSTOS VERA RHAY PABLO & 16 & 16 & 41.19 & 50.00 \%\\ \hline
Estructura de Datos & SAYAGO HEREDIA JAIME PAUL & 15 & 10 & 31.33 & 46.67 \%\\ \hline
Lectura y Escritura Académica & RAMIREZ LOZADA HAYDEE  & 16 & 16 & 45.81 & 68.75 \%\\ \hline
Arquitectura de Computadores & CARVAJAL CARVAJAL JOSE LUIS & 16 & 15 & 37.06 & 56.25 \%\\ \hline
Jesucristo y la Persona de Hoy & QUINTERO ROSALES FRANCISCO JHONNY & 16 & 14 & 33.88 & 56.25 \%\\ \hline
\end{tabularx}

\vspace{1cm}
\section{Análisis de Rendimiento}
Análisis del Rendimiento Académico del Curso de Primer Nivel de TICs

El presente análisis tiene como objetivo evaluar el rendimiento académico de los estudiantes de primer nivel de TICs en las diferentes materias que componen el currículum. A continuación, se presentan los resultados de dicha evaluación.

En general, se observa que la mayoría de las materias presentan un buen rendimiento académico, con un promedio superior al 40\%. Destaca la materia de "Lectura y Escritura Académica", que obtiene un promedio del 45.81\%, lo que indica que los estudiantes tienen un buen dominio de los principios de comunicación académica. Por otro lado, la materia de "Estructura de Datos" presenta un promedio más bajo, de 31.33\%, lo que sugiere que los estudiantes necesitan mejorar en este tema. Sin embargo, es importante destacar que aún presenta un porcentaje aceptable de aprobados, del 46.67\%.

En cuanto a la materia de "Redes I", se observa que los estudiantes han logrado un buen rendimiento, con un promedio del 35.81\% y un porcentaje de aprobados del 62.50\%. Esta buena performance puede deberse al interés y motivación de los estudiantes por el tema de redes.

La materia de "Arquitectura de Computadores" también presenta un buen rendimiento, con un promedio del 37.06\% y un porcentaje de aprobados del 56.25\%. En general, se puede decir que la mayoría de las materias presentan un buen rendimiento, aunque algunos temas requieren un esfuerzo adicional por parte de los estudiantes.

En relación con la materia de "Estadística y Probabilidades", se observa que los estudiantes han logrado un promedio algo más bajo, del 41.19\%, y un porcentaje de aprobados del 50\%. Esto puede deberse a la dificultad inherente del tema, que requiere un conocimiento previo y una comprensión profunda de los conceptos estadísticos.

Por último, en la materia de "Jesucristo y la Persona de Hoy", se observa un promedio del 33.88\% y un porcentaje de aprobados del 56.25\%. Aunque este resultado puede no ser tan favorable como otros, es importante considerar que este tema es más enfocado hacia la formación de la personalidad y el crecimiento espiritual que hacia la competitividad académica.

En conclusión, el análisis del rendimiento académico del curso de primer nivel de TICs revela que la mayoría de las materias presentan un buen rendimiento, aunque algunos temas requieren un esfuerzo adicional por parte de los estudiantes. Es importante que los docentes y los estudiantes trabajen juntos para mejorar los resultados y alcanzar los objetivos educativos del curso.\\
\vspace{1cm}\small
\begin{tabularx}{\textwidth}{|p{2.5cm}|p{2.5cm}|X|X|X|X|}
\hline
\multicolumn{6}{|X|}{\textbf{Nivel: 05 }}\\\hline\textbf{Materia} & \textbf{Docente} & \textbf{Estudiantes} & \textbf{Aprobados} & \textbf{Promedio} & \textbf{\%Supera el Promedio} \\ \hline
Redes Inalambricas & VELASTEGUI IZURIETA HOMERO JAVIER & 17 & 17 & 41.65 & 52.94 \%\\ \hline
Desarrollo Basado En Plataform & SAYAGO HEREDIA JAIME PAUL & 18 & 16 & 35.5 & 61.11 \%\\ \hline
Base de Datos 2 & CARVAJAL CARVAJAL JOSE LUIS & 17 & 17 & 41.82 & 58.82 \%\\ \hline
Administracion de Sistemas Ope & PLATA CABRERA CARLOS SIMON & 17 & 17 & 48.35 & 58.82 \%\\ \hline
Arquitectura y Plataforma de S & SAYAGO HEREDIA JAIME PAUL & 17 & 17 & 39.59 & 58.82 \%\\ \hline
Tutorías de Acompañamiento & CHILA GARCIA KAREN CAROLINA & 1 & 0 & 0 & 0.00 \%\\ \hline
Tutorías de Acompañamiento & VELASTEGUI IZURIETA HOMERO JAVIER & 71 & 0 & 0 & 0.00 \%\\ \hline
Etica Personal y Socioambienta & BAUTISTA COTERA JAVIER GEOVANNY & 17 & 17 & 45.71 & 70.59 \%\\ \hline
\end{tabularx}

\vspace{1cm}
\section{Análisis de Rendimiento}
Análisis del Rendimiento Académico de los Estudiantes de Primer Nivel de TICs

La presente información de rendimiento académico de los estudiantes de primer nivel de TICs presenta una variedad de resultados, reflejando la heterogeneidad de habilidades y conocimientos en diferentes materias. Entre las materias impartidas, destacan "Redes Inalambricas" y "Etica Personal y Socioambiental", que lograron promedios más altos, con un 52.94\% y un 70.59\% respectivamente.

En relación con el promedio general del curso, se observa que solo tres materias, "Redes Inalambricas", "Base de Datos 2" y "Etica Personal y Socioambiental", superaron este promedio, con porcentajes de aprobación del 52.94\%, 58.82\% y 70.59\% respectivamente. Por otro lado, materias como "Desarrollo Basado En Plataform" y "Administracion de Sistemas Operativos" presentaron resultados más moderados, con promedios del 35.5\% y 48.35\% respectivamente.

Es importante destacar que la materia de "Tutorías de Acompañamiento" presente un alto número de estudiantes matriculados, 71 en total, pero sin aprobados, lo que sugiere una necesidad de reajuste en la estrategia de enseñanza o apoyo para esta materia. Además, se observa que el número de matriculados varía según la materia, lo que podría ser un indicador de que los estudiantes se centran en áreas que les sean más ajenas o de menor interés.

En general, el análisis del rendimiento académico de los estudiantes de primer nivel de TICs muestra que, aunque hay materias que han logrado buenos resultados, también hay áreas que requieren mejoras. Es fundamental que los docentes diseñen estrategias efectivas para apoyar a los estudiantes y motivarlos para alcanzar un mejor rendimiento académico.\\
\vspace{1cm}\small
\begin{tabularx}{\textwidth}{|p{2.5cm}|p{2.5cm}|X|X|X|X|}
\hline
\multicolumn{6}{|X|}{\textbf{Nivel: 07 }}\\\hline\textbf{Materia} & \textbf{Docente} & \textbf{Estudiantes} & \textbf{Aprobados} & \textbf{Promedio} & \textbf{\%Supera el Promedio} \\ \hline
Gestion y Seguridad de Redes & VELASTEGUI IZURIETA HOMERO JAVIER & 13 & 13 & 42.54 & 46.15 \%\\ \hline
Practicas Pre Profesionales & CARVAJAL CARVAJAL JOSE LUIS & 9 & 9 & 48.56 & 88.89 \%\\ \hline
Prácticas de Servicio a la Com & SAYAGO HEREDIA JAIME PAUL & 14 & 14 & 44.57 & 64.29 \%\\ \hline
Interacción Humano Computador & PICO VALENCIA PABLO ANTONIO & 13 & 12 & 36.85 & 61.54 \%\\ \hline
Herramientas y Técnicas de Cib & VELASTEGUI IZURIETA HOMERO JAVIER & 13 & 13 & 37 & 30.77 \%\\ \hline
Diseño y Evaluación de Proyect & QUIÑONEZ KU VICTOR XAVIER & 13 & 13 & 37.46 & 46.15 \%\\ \hline
Integración Curricular & SINCHI SINCHI HUGO FERNANDO & 2 & 1 & 22 & 50.00 \%\\ \hline
\end{tabularx}

\vspace{1cm}
\section{Análisis de Rendimiento}
El análisis del rendimiento académico de los estudiantes de primer nivel de TICs revela varios tendencias y patrones que destacan la calidad y efectividad de las materias ofrecidas. En general, puede observarse que la mayoría de las materias tienen una tasa de aprobados elevada, con algunas excepciones.

La materia "Gestion y Seguridad de Redes" tiene un promedio general de 42.54 y un porcentaje de estudiantes que supera el promedio de 46.15\%. Esto sugiere que los estudiantes tienen una buena comprensión de los conceptos y habilidades relacionadas con la gestión y seguridad de redes.

En contraste, la materia "Herramientas y Técnicas de Cib" tiene un promedio general de 37 y un porcentaje de estudiantes que supera el promedio de 30.77\%, lo que indica que algunos estudiantes están experimentando dificultades para aprobar esta materia.

Otra materia que destaca es "Prácticas Pre Profesionales", con un promedio general de 48.56 y un porcentaje de estudiantes que supera el promedio de 88.89\%. Esto sugiere que los estudiantes están desarrollando habilidades prácticas y valiosas en el campo de las tecnologías de la información y comunicación.

La materia "Interacción Humano Computador" también destaca, con un promedio general de 36.85 y un porcentaje de estudiantes que supera el promedio de 61.54\%. Sin embargo, es importante notar que este promedio es algo más bajo que otros materiales.

Por otro lado, la materia "Diseño y Evaluación de Proyectos" tiene un promedio general de 37.46 y un porcentaje de estudiantes que supera el promedio de 46.15\%, lo que indica que esta materia también presenta algunas dificultades para los estudiantes.

En cuanto a la materia "Integración Curricular", se observa que tiene un promedio general bajo, con solo 22 puntos y un porcentaje de estudiantes que supera el promedio de 50.00\%. Esto sugiere que esta materia puede necesitar una revisión y ajuste para mejorar la comprensión y el rendimiento de los estudiantes.

En conclusión, el análisis del rendimiento académico de los estudiantes de primer nivel de TICs muestra generalmente buenos resultados, con algunas materias como "Gestion y Seguridad de Redes" y "Prácticas Pre Profesionales" que tienen buenos promedios y porcentajes de aprobados. Sin embargo, algunas materias como "Herramientas y Técnicas de Cib" y "Diseño y Evaluación de Proyectos" presentan dificultades para los estudiantes. Es importante continuar monitoreando el rendimiento y ajustar las estrategias de enseñanza para asegurarse de que los estudiantes tengan una comprensión adecuada de los conceptos y habilidades.\\
\vspace{1cm}\begin{tabularx}{\textwidth}{|X|X|X|}
\hline
\textbf{ELABORADO POR:} & \textbf{REVISADO POR:} & \textbf{APROBADO POR:} \\ \hline
Firma: & Firma: & Firma:\\
&&\\
&&\\
&&\\ \hline
\textbf{Nombre: Homero Velasteguí} & \textbf{Nombre: Manuel Nevarez} & \textbf{Nombre: Pablo Pico Valencia PhD.} \\ \hline
\textbf{Cargo: Coordinador Carrera} & \textbf{Cargo: Consejo de Escuela} & \textbf{Cargo: Director Académico} \\ \hline
\textbf{Fecha: 9/3/2024} & \textbf{Fecha: 9/3/2024} & \textbf{Fecha: 9/3/2024} \\ \hline
\end{tabularx}
