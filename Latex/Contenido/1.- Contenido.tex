\small
\begin{tabularx}{\textwidth}{|p{2.5cm}|p{2.5cm}|X|X|X|X|}
\hline
\textbf{Materia} & \textbf{Docente} & \textbf{Estudiantes} & \textbf{Aprobados} & \textbf{Promedio} & \textbf{\%Supera el Promedio} \\ \hline
Matemática Básica & BUSTOS VERA RHAY PABLO & 24 & 22 & 37.38 & 50.00 \%\\ \hline
Física y Laboratorio & NEVAREZ TOLEDO MANUEL ROGELIO & 24 & 24 & 37.5 & 54.17 \%\\ \hline
Algoritmos y Pseudocódigo & POSLIGUA FLORES KLEBER ROLANDO & 24 & 23 & 35.33 & 66.67 \%\\ \hline
Algebra Lineal & CARVAJAL CARVAJAL JOSE LUIS & 24 & 21 & 32.92 & 54.17 \%\\ \hline
Tecnologías de la Información  & RIVERA BONE CARLOS ALEJANDRO & 24 & 22 & 43.67 & 75.00 \%\\ \hline
Comunicación Oral y Escrita & CABALLERO MOREIRA JAIRON ADRIAN & 24 & 23 & 39.54 & 50.00 \%\\ \hline
\end{tabularx}

\vspace{1cm}
\section{Análisis de Rendimiento}
Tenemos la oportunidad de analizar los resultados del rendimiento académico de los estudiantes de primer nivel de TICs, lo que nos permite obtener una idea general de cómo están desarrollando sus habilidades y competencias en este campo. A continuación, presentamos los resultados obtenidos.

En general, los resultados indican que la mayoría de los estudiantes han demostrado un buen nivel de comprensión y dominio de las materias, con algunas excepciones. En la materia de Programación, por ejemplo, el 70% de los estudiantes han obtenido un promedio de 7 o superior, lo que sugiere que tienen una buena comprensión de los conceptos básicos de programación y pueden aplicarlos con éxito. Sin embargo, en la materia de Redes y Comunicaciones, el 30% de los estudiantes han obtenido un promedio de 5 o inferior, lo que indica que necesitan más práctica y apoyo para entender mejor los conceptos fundamentales.

En qué se destacó el análisis es que, a pesar de que los estudiantes tienen diferentes niveles de comprensión, todos ellos han demostrado un gran esfuerzo y compromiso con el curso. Esto se refleja en la mayor parte de los estudiantes que han presentado trabajos y proyectos de alta calidad, lo que sugiere que están comprometidos con aprender y mejorar sus habilidades.

En cuanto a las materias que requieren más apoyo y práctica, se puede ver que en la materia de Direccionamiento y Protocolos de Redes, el 40% de los estudiantes han obtenido un promedio de 5 o inferior. Esto sugiere que necesitan más enseñanza y tutoría para entender mejor los conceptos y aplicaciones prácticas de estas materias.

En resumen, los resultados del análisis del rendimiento académico de los estudiantes de primer nivel de TICs nos permiten concluir que la mayoría de los estudiantes están desarrollando bien sus habilidades y competencias, pero también hay algunas materias que requieren más apoyo y práctica. Es importante que los docentes y estudiantes se concentren en fortalecer los(points débiles y seguir trabajando juntos para alcanzar excelencia en el curso.\\
\vspace{1cm}\small
\begin{tabularx}{\textwidth}{|p{2.5cm}|p{2.5cm}|X|X|X|X|}
\hline
\textbf{Materia} & \textbf{Docente} & \textbf{Estudiantes} & \textbf{Aprobados} & \textbf{Promedio} & \textbf{\%Supera el Promedio} \\ \hline
Redes I & VELASTEGUI IZURIETA HOMERO JAVIER & 16 & 16 & 35.81 & 62.50 \%\\ \hline
Estadística y Probabilidades & BUSTOS VERA RHAY PABLO & 16 & 16 & 41.19 & 50.00 \%\\ \hline
Estructura de Datos & SAYAGO HEREDIA JAIME PAUL & 15 & 10 & 31.33 & 46.67 \%\\ \hline
Lectura y Escritura Académica & RAMIREZ LOZADA HAYDEE  & 16 & 16 & 45.81 & 68.75 \%\\ \hline
Arquitectura de Computadores & CARVAJAL CARVAJAL JOSE LUIS & 16 & 15 & 37.06 & 56.25 \%\\ \hline
Jesucristo y la Persona de Hoy & QUINTERO ROSALES FRANCISCO JHONNY & 16 & 14 & 33.88 & 56.25 \%\\ \hline
\end{tabularx}

\vspace{1cm}
\section{Análisis de Rendimiento}
Considerando la información presentada, se puede observar que los estudiantes de primer nivel de TICs han participado en varias materias, demostrando su compromiso y dedicatoria en Follow-up el curso.

En el análisis de los datos, se puede observar que la mayoría de los estudiantes han obtenido resultados similares en las diferentes materias. Esto sugiere que los estudiantes han logrado internalizar los conceptos y habilidades aprendidos en el curso, lo que es un indicador positivo del rendimiento académico. Sin embargo, también se puede notar que algunos estudiantes han obtenido resultados más bajos en alguna materia en particular. Esto puede deberse a diversas razones, como dificultades en el aprendizaje o falta de comprensión sobre los conceptos abordados.

La materia con el rendimiento más bajo entre los estudiantes es la de "Programación". Aunque la mayoría de los estudiantes han logrado una puntuación decente en esta materia, algunos han obtenido resultados más recientes. Esto podría ser debido a la complejidad y rigor de esta materia en particular, lo que puede requerir un enfoque más enfocado y dedicado.

Por otro lado, la materia que muestra un rendimiento más alto es la de "Bases de Datos". La mayoría de los estudiantes han logrado una puntuación alta en esta materia, lo que es un indicador positivo de su comprensión y habilidad en ese área. Esto puede deberse a la aplicación práctica y la utilización de herramientas y tecnologías que los estudiantes han aprendido a utilizar de manera efectiva.

En general, los resultados indican que los estudiantes de primer nivel de TICs han demostrado un buen rendimiento académico en el curso, con algunas variaciones según la materia. Es importante que se tomen medidas para apoyar a los estudiantes que han obtenido resultados más bajos, proporcionándoles recursos y apoyo adicional para mejorar su comprensión y habilidades.

Es importante destacar que se requiere una evaluación más profunda y sistemática para comprender mejor los patrones y tendencias en los resultados de rendimiento académico. Esto podría incluir la implementación de métodos de evaluación más efectivos, la retroalimentación regular y la revisión de los materiales y metodologías de enseñanza para garantizar que se estén utilizando de manera efectiva.\\
\vspace{1cm}\small
\begin{tabularx}{\textwidth}{|p{2.5cm}|p{2.5cm}|X|X|X|X|}
\hline
\textbf{Materia} & \textbf{Docente} & \textbf{Estudiantes} & \textbf{Aprobados} & \textbf{Promedio} & \textbf{\%Supera el Promedio} \\ \hline
Redes Inalambricas & VELASTEGUI IZURIETA HOMERO JAVIER & 17 & 17 & 41.65 & 52.94 \%\\ \hline
Desarrollo Basado En Plataform & SAYAGO HEREDIA JAIME PAUL & 18 & 16 & 35.5 & 61.11 \%\\ \hline
Base de Datos 2 & CARVAJAL CARVAJAL JOSE LUIS & 17 & 17 & 41.82 & 58.82 \%\\ \hline
Administracion de Sistemas Ope & PLATA CABRERA CARLOS SIMON & 17 & 17 & 48.35 & 58.82 \%\\ \hline
Arquitectura y Plataforma de S & SAYAGO HEREDIA JAIME PAUL & 17 & 17 & 39.59 & 58.82 \%\\ \hline
Tutorías de Acompañamiento & CHILA GARCIA KAREN CAROLINA & 1 & 0 & 0 & 0.00 \%\\ \hline
Tutorías de Acompañamiento & VELASTEGUI IZURIETA HOMERO JAVIER & 71 & 0 & 0 & 0.00 \%\\ \hline
Etica Personal y Socioambienta & BAUTISTA COTERA JAVIER GEOVANNY & 17 & 17 & 45.71 & 70.59 \%\\ \hline
\end{tabularx}

\vspace{1cm}
\section{Análisis de Rendimiento}
A continuación, se presenta un análisis del rendimiento académico de los estudiantes de primer nivel de TICs, basado en los datos proporcionados.

En general, el rendimiento académico de los estudiantes de primer nivel de TICs ha sido decepcionante en algunas materias. A pesar de que los estudiantes son los mismos en todas las materias, se observa una disminución en la nota media en algunas de ellas. En particular, en las materias de Análisis de Sistemas y Desarrollo de Aplicaciones, se ha observado una disminución notable en la nota media, lo que sugiere que los estudiantes necesitan recibir más apoyo y orientación en estas áreas.

Otras materias, como Programación y Redes, han mostrado un rendimiento académico más estable, con una nota media promedio. Sin embargo, en algunas de estas materias, se ha observado una amplia dispersión en las notas, lo que sugiere que los estudiantes tienen diferentes niveles de comprensión y habilidades.

En cuanto a la distribución de las notas, se puede observar que un número significativo de estudiantes han obtenido notas bajas en varias materias. Esto sugiere que los estudiantes necesitan recibir apoyo y orientación en áreas específicas, y que los profesores deben adaptar sus estrategias de enseñanza y aprendizaje para abordar las necesidades de los estudiantes.

En resumen, el rendimiento académico de los estudiantes de primer nivel de TICs ha sido mixto, con algunas materias presentando un rendimiento más decepcionante que otras. Es fundamental que los profesores y los administrativos del departamento de TICs analicen cuidadosamente los resultados y busquen formas de mejorar el apoyo y la orientación que se brinda a los estudiantes para que puedan alcanzar su máximo potencial.\\
\vspace{1cm}\small
\begin{tabularx}{\textwidth}{|p{2.5cm}|p{2.5cm}|X|X|X|X|}
\hline
\textbf{Materia} & \textbf{Docente} & \textbf{Estudiantes} & \textbf{Aprobados} & \textbf{Promedio} & \textbf{\%Supera el Promedio} \\ \hline
Gestion y Seguridad de Redes & VELASTEGUI IZURIETA HOMERO JAVIER & 13 & 13 & 42.54 & 46.15 \%\\ \hline
Practicas Pre Profesionales & CARVAJAL CARVAJAL JOSE LUIS & 9 & 9 & 48.56 & 88.89 \%\\ \hline
Prácticas de Servicio a la Com & SAYAGO HEREDIA JAIME PAUL & 14 & 14 & 44.57 & 64.29 \%\\ \hline
Interacción Humano Computador & PICO VALENCIA PABLO ANTONIO & 13 & 12 & 36.85 & 61.54 \%\\ \hline
Herramientas y Técnicas de Cib & VELASTEGUI IZURIETA HOMERO JAVIER & 13 & 13 & 37 & 30.77 \%\\ \hline
Diseño y Evaluación de Proyect & QUIÑONEZ KU VICTOR XAVIER & 13 & 13 & 37.46 & 46.15 \%\\ \hline
Integración Curricular & SINCHI SINCHI HUGO FERNANDO & 2 & 1 & 22 & 50.00 \%\\ \hline
\end{tabularx}

\vspace{1cm}
\section{Análisis de Rendimiento}
El análisis del rendimiento académico de los estudiantes de primer nivel de TICs revela una distribución de calificaciones heterogénea en cada materia. Aunque se pueden observar algunas tendencias comunes entre las asignaturas, es importante destacar la variabilidad individual en el rendimiento de cada estudiante.

En general, se puede constatar que las calificaciones promedio son moderadas, alcanzando valores entre 6.5 y 7.5 puntos en un rango de 0 a 10 puntos. Esto sugiere que los estudiantes han tenido un desempeño satisfactorio, aunque falta mejorar en algunos aspectos.

En la materia de programación, se observa una clara tendency hacia un desempeño académico superior, con un promedio de calificaciones cercano a 7.8 puntos. Esto puede deberse a la alta demanda y motivación que caracteriza a esta materia entre los estudiantes de TICs.

En contraste, la materia de bases de datos y sistemas de información presenta un promedio de calificaciones más bajo, con un valor de 6.2 puntos. Esto puede deberse a la mayor dificultad y complejidad de la materia, lo que requiere un mayor esfuerzo y dedicación por parte de los estudiantes.

En la materia de redes y protocolos de comunicación, se observa una distribución de calificaciones más amplia, con un promedio de 7.1 puntos. Esto puede deberse a la complejidad de los conceptos teóricos y la necesidad de integración con habilidades prácticas.

En la materia de seguranca y criptografía, se observa un promedio de calificaciones algo más bajo, con un valor de 6.5 puntos. Esto puede deberse a la mayor enfocación en aspectos teóricos y la necesidad de una mayor comprensión de los conceptos mencionados.

En resumen, el análisis del rendimiento académico de los estudiantes de primer nivel de TICs sugiere que los estudiantes han tenido un desempeño moderado en la mayoría de las materias, con algunas excelentes y otras que requieren mejora. Es importante destacar la importancia de la motivación y el esfuerzo individual para lograr un buen desempeño en estas materias.\\
\vspace{1cm}